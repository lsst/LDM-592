\documentclass[DM,toc,lsstdraft]{lsstdoc}

\setDocChangeRecord{%
\addtohist{1.0}{2017-??-??}{Initial version.} {T.~Jenness}
}

\title{Data Access Use Cases}
\date{\today}

\author{Tim~Jenness, Jim~Bosch, Michelle~Gower, Simon~Krughoff,
Russell~Owen, Pim~Schellart, Brian~van~Klaveren, Dominique~Boutigny}
\setDocRef      {LDM-592} % the reference code
\setDocUpstreamLocation{\url{https://github.com/lsst/LDM-592}}
%
% a short abstract
%
\setDocAbstract{Use Cases written by the Butler Working Group covering data discovery, data storage, and data retrieval.}

\begin{document}
%
% the title page
%
\maketitle

\section{Introduction}

The Butler Working Group \citedsp{LDM-563} was convened in August 2017 to work on requirements and design of the Butler.
As part of the process, Use Cases were written to anchor the requirements.
In this document we present those Use Cases, grouped by the originating team, with labels that allow the requirements in \citeds{LDM-556} to reference them.

\section{Actors}

The use cases make use of a standard set of Actors.

\subsection{General}

\begin{description}
\item[Observer] Observatory Operations (e.g., summit, base)
\item[Astronomer] End User with data rights to releases
\item[Data Quality Assessment Board] Board responsible for the quality of data releases, including approval of software versions and flagging of data quality.
\end{description}

\subsection{Staff Scientist or Staff Developer}

\begin{description}
\item[Data Facility Scientist] Data Facility Scientist (leadership?)
\item[Pipelines Developer] Science Operations Scientist (algorithm developer)
\item[QA Scientist] Science Operations Scientist (long term QA)
\item[QA Collaborator] Collaborator with Science OPS (non-LSST funded)
\item[Developer] Non-science LSST-funded developer (e.g., QSERV, Batch Processing Service, LSP)
\item[Batch Production Operator] Production Operator for DRP (has additional privileges)
\item[Commissioning Scientist] Scientist involved in commissioning (has additional privileges)
\end{description}

\subsection{Software Agents}

\begin{description}
\item[SuperTask] Software agent that takes input files, processes them, and supplies output files.
\item[CI System] Software agent that runs tests (including integration tests that may involve pipeline processing) on a regular cadence
\item[Batch Control System] Software agent that manages SuperTask execution running in the batch production environment
\item[Harvesting Agent] Scans the file system when a job completes and stores the resulting files in the Data Backbone.
\item[DMCS] Data Management Control System
\end{description}

\section{Architecture}



\bibliography{local,lsst,lsst-dm,refs,books,refs_ads}

\end{document}
